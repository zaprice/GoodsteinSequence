% Goodstein's Theorem as a Natural Independence Phenomenon:A Mathematical Adventure, by Zachary A. Price
% This work is licensed under the Creative Commons Attribution-ShareAlike 4.0 International License. To view a copy of this license, visit http://creativecommons.org/licenses/by-sa/4.0/.

\documentclass[10pt]{article}

\usepackage{amsmath,amssymb, amsthm}
\usepackage[margin=1in]{geometry}
\usepackage{graphicx}
\usepackage{ccicons}
\usepackage{hyperref}

\setlength{\parindent}{0in}
\setlength{\parskip}{1em}

\newcommand{\Z}{\mathbb{Z}}
\newcommand{\N}{\mathbb{N}}
\newcommand{\R}{\mathbb{R}}
\newcommand{\epz}{\varepsilon_0}

\title{Goodstein's Theorem as a Natural Independence Phenomenon:\newline A Metamathematical Adventure\footnote {
TeX for this document is available at \url{https://github.com/zaprice}
}}
\author{Zachary Price\footnote { 
\ccbysa \indent This work is licensed under the Creative Commons Attribution-ShareAlike 4.0 International License. To view a copy of this license, visit http://creativecommons.org/licenses/by-sa/4.0/
}}

\begin{document}
\maketitle

\newtheorem{define}{Definition}
\newtheorem{thm}{Theorem}
\newtheorem{lem}{Lemma}
\newtheorem{conj}{Conjecture}

\begin{abstract}
Goodstein's theorem is an example of a natural independence phenomenon, a theorem which is intuitively "true" yet unprovable under a particular set of axioms due to incompleteness.
This theorem serves as an illustration of the limitations of the models we use to reason formally about mathematical objects.
We will investigate the connection between G{\"o}del's incompleteness theorems and Goodstein's theorem by exploring a few areas of mathematics, including proof theory and ordinal numbers.
This mathematical adventure is batteries-included (i.e. with proofs).
\end{abstract}

\section{Introduction}
This section will contain introductory information about our topics, including a description and statement of Goodstein's theorem.

G{\"o}del's incompleteness theorems are a touchstone work of mathematical logic. Incompleteness has powerful effects across all of fundamental mathematics.
We are very familiar with some of the big implications of incompleteness, like the independence of the continuum hypothesis from Zermelo-Fraenkel set theory.
Such results are informally referred to as natural independence phenomena: examples where the unprovability of a statement arises, under certain axioms, as a result of incompleteness.
In this paper, we will consider a theorem of Reuben Goodstein which is naturally independent from the axioms of Peano arithmetic.
Goodstein's theorem is of particular interest as it is a very simple example of independence; the statement of the theorem involves only sequences of integers that terminate in finite time.
Yet, the metamathematical results surrounding the theorem touch on very deep principles in mathematics.

To state Goodstein's theorem, we will need the notion of hereditary base-$n$ notation.

\begin{define}
We say $m\in\N$ is written in \emph{hereditary base-$n$} if $m$ is written as:
$$m = \sum_{i=0}^{k} a_in^i = a_kn^k + a_{k-1}n^{k-1} + ... + a_1n + a_0$$
where $n>1$ and $0\leq a_i < n$	for all $a_i$, with $a_k>0$.
\end{define}

\begin{define}
We define $B[b]$, \emph{base change operator} in base $b$, as follows:
\end{define}


\begin{define}
The \emph{Goodstein sequence with seed $m$}, $G(m)$, is defined recursively as follows:
$G_1(m) = m$
\end{define}

% With a few examples, we will demonstrate the intuitive notion of why this theorem is true

% Indeed, given enough time and a universe that allows, one could compute the length of any particular Goodstein sequence; this does not necessarily mean we can prove that they all terminate.

\section{Ordinal Numbers}
This section will contain an introduction to ordinal numbers and ordinal arithmetic; these techniques are necessary to prove Goodstein's theorem in ZFC.

\section{Proof of Goodstein's Theorem}
This section will contain a proof of Goodstein's theorem, using the techniques developed in the previous section.

\section{Peano Arithmetic}
This section will contain an introduction to axiomatic systems, particulary Peano arithmetic, with a focus on the (informal) relative strength of different systems.
We will be careful to note how Goodstein's theorem is stated in the language of Peano arithmetic.

\section{G{\"o}del's Incompleteness Theorems}
This section will contain an informal description of G{\"o}del's incompleteness theorems, including their significance to mathematics in general.
Then, we will follow with a likely-abridged, but mostly formal, retelling of G{\"o}del's proofs.

\section{Gentzen's Consistency Proof}
This section will contain Gerhard Gentzen's proof of the consistency of Peano arithmetic, using transfinite induction to $\epz$.
We will be careful to note how this does not contradict G{\"o}del's work, and the further implications about the problem at hand.

\section{Kirby and Paris}
This section will contain the proof of Laurence Kirby and Jeff Paris, showing that Gentzen's theorem implies the independence of Goodstein's theorem from Peano arithmetic.

\section{Epilogue}
In this section, we will conclude our mathematical adventure and reflect on what we have learned.

\section{Appendix I}
This appendix will contain a more formal detour into proof theory, mostly exploring the concept of the proof-theoretic ordinal.

\section{Appendix II}
This appendix will contain a detour into fast-growing hierarchies, showing interesting connections between computational complexity, the ordinals, and the analysis contained in Appendix I.

\end{document}